%! Author = julianmour
%! Date = 01/05/2023

\section{Preliminary Results}
We evaluated our preliminary approach on the MNIST dataset, consisting of images showing a singular digit.
The classifier's goal is to return the correct classification of a given diget. % from ten classes;
%$C = \{0, 1, \ldots,9\}$ (the 10 different digits), each image classified to two different classes (contains two different digits).
%%%%%An example of an image is shown in Figure~\ref{fig:double-mnist-sample}. %, where the digit $4$ is the target object and the digit $9$ is the non-target object.
We ran our algorithm on three different models: a fully connected network 3X10, and two convolutional neural networks with different number of iterations during training. We encoded each of those models and the problem definition into MIPVerify ~\cite{MIPVERIFY} which returns the bounds for the confidence and the time it took to solve. The timeout was set to 50800 seconds. We compared between two approaches for each model:
\begin{itemize}
    \item Solving the confidence bounds without the proposed constraints approach.
    \item Solving the confidence bounds using the proposed constraints approach.
\end{itemize}
The preliminary results show a significant improvement of 82\% in execution time when applying the second approach.
The bounds for the confidence were similar in both of the approaches.