%! Author = julianmour
%! Date = 01/05/2023

\section{Preliminary Results}
In this section, we describe our preliminary results.
We evaluate our approach on classifiers for the MNIST dataset, consisting of images showing a single digit.
The classifier's goal is to return the digit shown in the image. In our experiments, we focus on the global robustness property for the class $c=0,1,2$ and.

We consider two network architectures: 
a fully connected network consisting of three layers, ten neurons in each, denoted \texttt{3x10}, and a convolutional neural network. 
Table \ref{table_architectures} describes our models. For each model $F$, we define $F_1$ as $F$ and $F_2$ as $F$ with a slight change in a single weight in the last layer on $F$. Specifically we add a small addition to the weight between neuron $1$ in layer $L-1$ to neuron $3$ in layer $L$. We evaluate the algorithm across different weight additions. 
Furthermore we compare between 3 different approaches to show the scalability of our approach:
\begin{itemize}
    \item Baseline: the naive approach.
    \item VHAGaR extension.
    \item Our approach.
\end{itemize} 
We provide the MILP solver a timeout of 40800 seconds.

\begin{table}[H]
    \centering
    \resizebox{\textwidth}{!}{
    \begin{tabular}{@{\extracolsep{\fill}}llll@{}}
        \toprule
        \makecell{Dataset} & \makecell{Name} & \makecell{Architecture}  & \makecell{Iterations during training} \\
        \midrule            
        \multirow{2}{*}{MNIST} & 3 x 10 & 3 fully-connected layers & 20 \\
                               & CNN & 2 convolutional (stride 4) and 2 fully-connected layers & 20 \\
                               %& CNN2 & 2 convolutional (stride 4) and 2 fully-connected layers & 19 \\
        \bottomrule
    \end{tabular}
    }
    \caption{The networks used for this experience.
        \label{table_architectures}}
\end{table}

The examined perturbation is called $L_\infty([0,\epsilon])$, where for a given $\epsilon\in{(0,1]}$, and the input is perturbed by changing the value of each pixel up by $|\epsilon|$.

Tables ~\ref{table_3_x_10,table_cnn0_1} show the results of all three approaches in the execution time for the 3X10 and CNN respectively.
Results show that our approach significant the execution time up to \Neta{FILL IN} compared to the baseline, and up to \Neta{FILL IN} compared to the VHAGaR Extension, while maintaining the accuracy of $\delta_c$ upper and lower bounds. In some cases, our bounds are tighter due to early stopping of the two other approaches due to timeout, by \Neta{FILL IN}
Namely, preliminary results show that our approach enhances scalability and consequently its accuracy.

\begin{table}[H]
  \centering
  \resizebox{\textwidth}{!}{
  \begin{tabular}{@{\extracolsep{\fill}}cccccccccccccc@{}}
    \toprule
    \multirow{3}{*}{} & \multirow{3}{*}{Class} & \multicolumn{3}{c}{$L_\infty$[0.1]} & \multicolumn{3}{c}{$L_\infty$[0.1]} & \multicolumn{3}{c}{$L_\infty$ [0.05]} & \multicolumn{3}{c}{$L_\infty$ [0.05]}\\
    & & \multicolumn{3}{c}{Weight addition[0.1]} & \multicolumn{3}{c}{Weight addition[0.05]} & \multicolumn{3}{c}{Weight addition[0.1]} & \multicolumn{3}{c}{Weight addition[0.05]}\\
    \cmidrule(lr){3-14}
    & & \makecell{time} & \makecell{$\delta_U$} & \makecell{$\delta_L$} & \makecell{time} & \makecell{$\delta_U$} & \makecell{$\delta_L$} & \makecell{time} & \makecell{$\delta_U$} & \makecell{$\delta_L$} & \makecell{time} & \makecell{$\delta_U$} & \makecell{$\delta_L$} \\
    \midrule
    \multirow{3}{*}{Baseline}  & Class 0 & 114.2 & 17.18 & 17.04 & 120.13 & 17.18 & 17.04 & 201.41 & 10.34 & 10.25 & 223.3 & 10.3 & 10.2 \\
 & Class 1 & 71.76 & 16.09 & 16.09 & 132.22 & 16.09 & 16.09 & 585.05 & 8.21 & 8.13 & 906.67 & 8.21 & 8.13 \\
 & Class 2 & 111.22 & 19.13 & 18.97 & 121.79 & 19.13 & 18.97 & 356.76 & 9.61 & 9.52 & 370.38 & 9.52 & 9.52 \\
\cmidrule{2-14}
\multirow{3}{*}{VHAGaR Extension}  & Class 0 & 59.6 & 17.13 & 17.13 & 80.77 & 17.27 & 17.13 & 134.84 & 10.34 & 10.25 & 145.57 & 10.34 & 10.25 \\
 & Class 1 & 91.07 & 16.09 & 16.09 & 78.65 & 16.24 & 16.09 & 316.54 & 8.22 & 8.14 & 257.23 & 8.22 & 8.14 \\
 & Class 2 & 75.28 & 19.18 & 19.02 & 73.86 & 19.18 & 19.02 & 221.37 & 9.6 & 9.52 & 215.08 & 9.6 & 9.52 \\
\cmidrule{2-14}
\multirow{3}{*}{Our approach}  & Class 0 & 60.97 & 17.27 & 17.13 & 75.63 & 17.28 & 17.13 & 118.74 & 10.28 & 10.18 & 133.44 & 10.32 & 10.24 \\
 & Class 1 & 61.69 & 16.23 & 16.09 & 62.43 & 16.21 & 16.06 & 267.57 & 8.21 & 8.14 & 244.15 & 8.21 & 8.14 \\
 & Class 2 & 63.14 & 19.14 & 18.97 & 65.87 & 19.2 & 19.02 & 160.3 & 9.6 & 9.52 & 148.26 & 9.61 & 9.52 \\
    \bottomrule
  \end{tabular}
  }

\caption{The lower and upper bounds for 3X10 architecture 
		\label{table_3_x_10}}
\end{table}


\begin{table}[H]
  \centering
  \resizebox{\textwidth}{!}{
  \begin{tabular}{@{\extracolsep{\fill}}cccccccccccccc@{}}
    \toprule
    \multirow{3}{*}{} & \multirow{3}{*}{Class} & \multicolumn{3}{c}{$L_\infty$[0.1]} & \multicolumn{3}{c}{$L_\infty$[0.1]} & \multicolumn{3}{c}{$L_\infty$ [0.08]} & \multicolumn{3}{c}{$L_\infty$ [0.08]}\\
    & & \multicolumn{3}{c}{Weight addition[0.1]} & \multicolumn{3}{c}{Weight addition[0.05]} & \multicolumn{3}{c}{Weight addition[0.1]} & \multicolumn{3}{c}{Weight addition[0.05]}\\
    \cmidrule(lr){3-14}
    & & \makecell{time} & \makecell{$\delta_U$} & \makecell{$\delta_L$} & \makecell{time} & \makecell{$\delta_U$} & \makecell{$\delta_L$} & \makecell{time} & \makecell{$\delta_U$} & \makecell{$\delta_L$} & \makecell{time} & \makecell{$\delta_U$} & \makecell{$\delta_L$} \\
    \midrule
    \multirow{3}{*}{Baseline}  & Class 0 & 59.82 & 29.07 & 29.01 & 36.88 & 29.07 & 28.97 & 155.33 & 29.06 & 28.84 & 49.59 & 29.07 & 28.94 \\
 & Class 1 & 5344.62 & 71.22 & 70.51 & 3264.78 & 71.2 & 70.5 & 40800.15 & 58.87 & 57.38 & 40800.07 & 59.1 & 57.38 \\
 & Class 2 & 40.66 & 22.46 & 22.46 & 12.88 & 22.46 & 22.46 & 25.16 & 22.46 & 22.45 & 14.32 & 22.46 & 22.43 \\
\cmidrule{2-14}
\multirow{3}{*}{VHAGaR Extension}  & Class 0 & 10.57 & 29.07 & 28.97 & 13.37 & 29.07 & 29.01 & 18.47 & 29.07 & 28.84 & 48.59 & 29.07 & 28.93 \\
 & Class 1 & 224.86 & 71.19 & 70.51 & 464.66 & 71.18 & 70.49 & 24136.86 & 57.95 & 57.38 & 23504.24 & 57.95 & 57.38 \\
 & Class 2 & 7.02 & 22.46 & 22.46 & 6.41 & 22.46 & 22.46 & 5.82 & 22.46 & 22.46 & 6.38 & 22.46 & 22.31 \\
\cmidrule{2-14}
\multirow{3}{*}{Our approach}  & Class 0 & 13.45 & 29.07 & 29.02 & 15.63 & 29.07 & 29.02 & 11.07 & 29.07 & 28.93 & 10.72 & 29.07 & 28.94 \\
 & Class 1 & 213.35 & 71.17 & 70.47 & 230.26 & 71.18 & 70.49 & 10800.09 & 58.09 & 57.38 & 10800.08 & 57.98 & 57.38 \\
 & Class 2 & 3.08 & 22.46 & 22.29 & 3.56 & 22.46 & 22.46 & 3.5 & 22.46 & 22.27 & 3.59 & 22.46 & 22.4 \\
    \bottomrule
  \end{tabular}
  }
\caption{The lower and upper bounds for CNN architecture with 20 iteration during training process.
		\label{table_cnn0_1}}
\end{table}

\begin{comment}

\begin{table}[H]
	\centering
    \resizebox{\textwidth}{!}{
	\begin{tabular}{@{\extracolsep{\fill}}llllll@{}}
		\toprule
		& \makecell{ } & \makecell{Brightness[0.1] \\ Weight addition[0.1]} & \makecell{Brightness[0.1] \\ Weight addition[0.05]} & \makecell{Brightness [0.05] \\ Weight addition[0.1]} & \makecell{Brightness [0.05] \\ Weight addition[0.05]}\\
		\midrule			
		\multirow{2}{*}{Base approach} & upper bound  & 15.77 & 17.71 & 15.6 & 19.16\\
                              & lower bound & 14.18 & 14.18 & 11.81 & 11.73\\
		\midrule
		\multirow{2}{*}{Our approach} & upper bound & 14.29 & 14.31 & 11.92 & 11.92\\
                              & lower bound & 14.15 & 14.17 & 11.8 & 11.81\\				
		\bottomrule
	\end{tabular}
}
\caption{The lower and upper bounds for CNN architecture with 19 iteration during training process.
		\label{table_cnn0_2}}
\end{table}

\end{comment}


% NETA - image of database + explain: "hey we did good - but not good enough". 

% bounds in table
% times show in graph 