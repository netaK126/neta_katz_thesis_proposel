\begin{comment}
\documentclass[11pt]{article}
\usepackage[a4paper, portrait, margin=1in]{geometry}
\usepackage{graphicx}
\usepackage{hyperref}
\usepackage{caption}
\usepackage[labelformat=simple]{subcaption}    %%Adding option to remove parenthesis
\renewcommand{\thesubfigure}{\normalsize Figure \thefigure. (\alph{subfigure}):}
% \usepackage{subcaption}
% \usepackage{subcaption}
%\usepackage{dirtytalk}
\usepackage{cjhebrew}
\usepackage{float}
\usepackage{eldar_report}
% Useful packages
\usepackage{amsmath}
\usepackage{amsfonts}
% \newcommand\norm[1]{\lVert#1\rVert}
\newcommand\normx[1]{\Vert#1\Vert}
\usepackage{graphicx}
\usepackage{enumitem}
\usepackage{lifetime}
\usepackage{booktabs}
\usepackage{makecell}
\usepackage{graphicx}
\usepackage{multirow}
\usepackage{comment}
\usepackage{array}
\usepackage[english]{babel}
\usepackage{amsthm}
\usepackage{float}
\newcommand{\Dana}[1]{\textcolor{purple}{\bf Dana: #1}}
\newcommand{\Neta}[1]{\textcolor{purple}{\bf Neta: #1}}

\begin{document}
\end{comment}

\newtheorem{theorem}{Theorem}[section]
\newtheorem{lemma}[theorem]{Lemma}
\begin{lemma}\label{lem:functionTheory}
Every function k(x) has a function g(x) such that $\forall{x}:$ $k(x)>g(x)$.
\end{lemma}

\begin{lemma}
$\delta_1^{opt} \geq \delta^{vhagar}_1$
\end{lemma}
\begin{proof}
%To prove it by contradiction try and assume that the statement is false, i.e. $\delta_1^{opt} < \delta^{vhagar}_1$.
Some new notions:
$$I^{vhagar} = \{(x,\epsilon)|\delta_1^{vhagar} \geq C(x,c,N_1) > 0 , C(f_P(x,\epsilon),c,N_1)\leq 0 \}$$
%$$I^{\delta'} = \{(x,\epsilon)|\delta^{max}>\delta' \geq C(x,c,N_1) > 0 , C(f_P(x,\epsilon),c,N_1)\leq 0 \} $$
$$I = \{(x,\epsilon)|\forall{x,\epsilon}:\ f_P(x,\epsilon)\in{I}\}$$
Therefore, $\forall{(x,\epsilon)}\in{I^{vhagar}}:$ $(x,\epsilon)\in{I}$.%, and specifically $(x^{vhagar},\epsilon^{vhagar})\in{I^{\delta'}}$, when $\delta'\geq{C(x^{vhagar},c,F_1)}$
$$I^{\delta'}_g = \{(x,\epsilon)|\delta^{max}> C(x,c,N_1) \geq \delta' > 0 ,\ C(f_P(x,\epsilon),c,N_1)\leq g(\delta',\epsilon), g(\delta',\epsilon)>0 \} $$
Therefore, $\forall{(x,\epsilon)}\in{I^{\delta'}}:$ $(x,\epsilon)\in{I}$.\\
We want to prove that $\forall{\delta'}: \delta^{max} > \delta' > \delta^{vhagar}$ we can find $x,\epsilon\in{I^{\delta'}_g}$ that satisfy: $\delta^{vhagar}<\delta'\leq{C(x,c,N_1)}$ and $0<C(f_P(x,\epsilon),c,N_1)<g(\delta',\epsilon)$.\\
prove by contradiction: Assuming that there is no input $x$ that satisfy $\delta^{vhagar}<\delta'<C(x,c,F_1)$ means that $\delta^{vhagar}=\delta^{max}$ which is a contradiction to $\delta^{vhagar}<\delta'<\delta^{max}$ or it contradicts the fact that we are looking at the entire input space, since otherwise $\delta^{vhagar}=\delta^{max}$ which is (again) a contradiction.\\
Then, we need to prove that for this $x$ exists such $\epsilon$ that satisfy: $0<C(f_P(x,\epsilon),c,N_1)<g(\delta',\epsilon)$.\\
First we prove that $0<C(f_P(x,\epsilon),c,N_1)$. If this statement was false, which means $0 \geq C(f_P(x,\epsilon),c,N_1)$, it would contradict Vhagar (since we found $\delta'>\delta^{vhagar}$ that satisfy the proposed problem by vhagar). than we choose a function $g$ that satisfy $C(f_P(x,\epsilon),c,N_1)<g(\delta',\epsilon)$, which must exists according to lemma \ref{lem:functionTheory}.\\
Specifically, it true for $\delta'=\delta_1^{OPT}$, and $g(\delta',\epsilon)=\frac{\delta_1^{opt}}{10}$ in our OPT problem.\\
Overall, we proved that $\forall{\delta'>\delta^{vhagar}}$ $\exists{x,\epsilon}$ such that $C(x,c,N_1)\geq{\delta'}$ (otherwise $\delta'=\delta^{max}$ which is a contradiction) and $C(f_P(x,\epsilon),c,N_1)\leq{g(\delta',\epsilon)}$

\end{proof}

\begin{lemma}
$\delta_2^{opt} \geq \delta^{vhagar}_2$
\end{lemma}
\begin{proof}
%To prove it by contradiction try and assume that the statement is false, i.e. $\delta_1^{opt} < \delta^{vhagar}_1$.
Let's examine the following optimization problem:
$$ max_x{\Delta}:\ 0 \leq |C(x,c,N_2)-C(x,c,N_1)|=\Delta $$
Which means that $\forall{x}:\ |C(x,c,N_2)-C(x,c,N_1)| \leq \Delta$.\\ \\
If $\Delta=0$ it means that $\forall{x}:\ |C(x,c,N_2)-C(x,c,N_1)|=0$, and specifically $\delta_1^{opt}=\delta_2^{opt}$. From previous lemma we can say $\delta_1^{opt} \geq \delta^{vhagar}_1$, so in that case, $\delta_2^{opt} \geq \delta^{vhagar}_1$. because of $\forall{x}:\ |C(x,c,N_2)-C(x,c,N_1)|=0$, we can say $\delta^{vhagar}_1=\delta^{vhagar}_2$ (otherwise it contradicts vaghar or the equality of the confidence-diff to 0). therefore: $\delta_2^{opt} \geq \delta^{vhagar}_2$ when $\Delta=0$.\\
We shall use the following notion: when $\Delta_{max}=0: \delta'=\delta_2^{opt}$ \\ \\
If $\Delta>0$: since $\forall{x}:\ |C(x,c,N_2)-C(x,c,N_1)| \leq \Delta$ it also happens:  
$$\forall{x}\in{\{x| C(x,c,N_2)\geq{{\delta_2^{opt}}_{\Delta=0}}\ C(x,c,N_1)\leq{\delta_1^{opt}}\}}: |C(x,c,N_2)-C(x,c,N_1)| \leq \Delta$$
we can also define:
$$ I_{\Delta\geq 0}= \{x| C(x,c,N_2)\geq{\delta'},\ C(x,c,N_1)\leq{\delta_1^{opt},\ |C(x,c,N_2)-C(x,c,N_1)| \leq \Delta,\ 0 \leq \Delta}\}$$
$$ I_{\Delta > 0}= \{x| C(x,c,N_2)\geq{\delta'},\ C(x,c,N_1)\leq{\delta_1^{opt},\ |C(x,c,N_2)-C(x,c,N_1)| \leq \Delta,\ 0 < \Delta}\}$$
$$ I_{\Delta=0}= \{x| C(x,c,N_2)\geq{\delta'},\ C(x,c,N_1)\leq{\delta_1^{opt},\ |C(x,c,N_2)-C(x,c,N_1)| \leq \Delta=0}\}$$
We shall notice that $I_{\Delta>0} \ne |I_{\Delta=0}|$ and therefore $I_{\Delta > 0} \ne \emptyset$.\\ \\
We can say that:
$$ max_{x\in{I_{\Delta=0}}}{C(x,c,N_2)}\ <\  max_{x'\in{I_{\Delta>0}}}{C(x',c,N_2)}$$
because  $|C(x',c,N_2)-C(x',c,N_1)|>0$ which implied that the confidence $C(x',c,N_2)$ is bigger: if $C(x',c,N_2)-C(x',c,N_1)>0$ than it is trivial. if $C(x',c,N_2)-C(x',c,N_1)<0$ than it is \Neta{TODO} .\\
for $x'$ we can say $C(x',c,N_2) \geq {\delta_2^{opt}}_{\Delta>0}$ (othwise it says that ${\delta_2^{opt}}_{\Delta>0}=\delta_2^{max}$). \\
therefore, ${\delta_2^{opt}}_{\Delta>0} \geq {\delta_2^{opt}}_{\Delta=0}\geq{\delta_2^{vhagar}}$




\begin{comment}
We can say that  $I_{\Delta}=\{x| C(x,c,N_2)\geq{{\delta_2^{opt}}_{\Delta=0}}\}\ \bigwedge\ \{x| C(x,c,N_1)\leq{\delta_1^{opt}}\}$. \\
However, the second group ($\{x| C(x,c,N_1)\leq{\delta_1^{opt}}\}$) is the same for $I_{\Delta>0}$ and for $I_{\Delta=0}$. Therefore:
$$ I_{\Delta\geq 0}^2=\{x| C(x,c,N_2)\geq{{\delta_2^{opt}}_{\Delta=0}}\}_{\Delta>0}  \supset I_{\Delta=0}^2=\{x| C(x,c,N_2)\geq{{\delta_2^{opt}}_{\Delta=0}}\}_{\Delta=0}$$
Which means that $\forall{x}\in{I_{\Delta\geq 0}^2-I_{\Delta=0}^2}: C(x,c,N_2) \geq C$
\end{comment}
\end{proof}

%\end{document} 