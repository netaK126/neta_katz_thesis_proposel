\begin{comment}
\documentclass[11pt]{article}
\usepackage[a4paper, portrait, margin=1in]{geometry}
\usepackage{graphicx}
\usepackage{hyperref}
\usepackage{caption}
\usepackage[labelformat=simple]{subcaption}    %%Adding option to remove parenthesis
\renewcommand{\thesubfigure}{\normalsize Figure \thefigure. (\alph{subfigure}):}
% \usepackage{subcaption}
% \usepackage{subcaption}
%\usepackage{dirtytalk}
\usepackage{cjhebrew}
\usepackage{float}
\usepackage{eldar_report}
% Useful packages
\usepackage{amsmath}
\usepackage{amsfonts}
% \newcommand\norm[1]{\lVert#1\rVert}
\newcommand\normx[1]{\Vert#1\Vert}
\usepackage{graphicx}
\usepackage{enumitem}
\usepackage{lifetime}
\usepackage{booktabs}
\usepackage{makecell}
\usepackage{graphicx}
\usepackage{multirow}
\usepackage{comment}
\usepackage{array}
\usepackage[english]{babel}
\usepackage{amsthm}
\usepackage{float}
\newcommand{\Dana}[1]{\textcolor{purple}{\bf Dana: #1}}
\newcommand{\Neta}[1]{\textcolor{purple}{\bf Neta: #1}}

\begin{document}
\end{comment}

\newtheorem{theorem}{Theorem}[section]
\newtheorem{lemma}[theorem]{Lemma}
\begin{lemma}\label{lem:functionTheory}
Every function k(x) has a function g(x) such that $\forall{x}:$ $k(x)>g(x)$.
\end{lemma}

\begin{lemma}
$\delta_1^{opt} \geq \delta^{vhagar}_1$
\end{lemma}
\begin{proof}
%To prove it by contradiction try and assume that the statement is false, i.e. $\delta_1^{opt} < \delta^{vhagar}_1$.
Some new notions:
$$I^{vhagar} = \{(x,\epsilon)|\delta_1^{vhagar} \geq C(x,c,N_1) > 0 , C(f_P(x,\epsilon),c,N_1)\leq 0 \}$$
%$$I^{\delta'} = \{(x,\epsilon)|\delta^{max}>\delta' \geq C(x,c,N_1) > 0 , C(f_P(x,\epsilon),c,N_1)\leq 0 \} $$
$$I = \{(x,\epsilon)|\forall{x,\epsilon}:\ f_P(x,\epsilon)\in{I}\}$$

Therefore, $\forall{(x,\epsilon)}\in{I^{vhagar}}:$ $(x,\epsilon)\in{I}$.%, and specifically $(x^{vhagar},\epsilon^{vhagar})\in{I^{\delta'}}$, when $\delta'\geq{C(x^{vhagar},c,F_1)}$
$$I^{\delta'}_g = \{(x,\epsilon)|\delta^{max}> C(x,c,N_1) \geq \delta' > 0 ,\ C(f_P(x,\epsilon),c,N_1)\leq g(\delta',\epsilon), \forall{\delta',\epsilon}:\ g(\delta',\epsilon)>0 \} $$
Therefore, $I^{vaghar}\subseteq{I^{\delta'}_g}$, which means:
$$ \delta_1^{opt}=max_{x\in{I^{\delta'}_g}}{C(x,c,N_1)} \geq max_{x\in{I^{vaghar}}}{C(x,c,N_1)}=\delta_1^{vhagar}$$
\end{proof}

\begin{lemma}
$\delta_2^{opt} \geq \delta^{vhagar}_2$
\end{lemma}
\begin{proof}
%To prove it by contradiction try and assume that the statement is false, i.e. $\delta_1^{opt} < \delta^{vhagar}_1$.
Previously we proved that $\delta_2^{\text{vhagar}} \leq \delta_1^{opt}+\delta_{\text{diff}}+\Delta$.\\
\begin{enumerate}[nosep,nolistsep]
    \item If $\delta_1^{opt}+\delta_{\text{diff}}+\Delta \leq \delta_2^{max}$:
    $$\exists{x}:\ C(x,c,N_2) \geq \delta_1^{opt}+\delta_{\text{diff}}+\Delta$$
    we want to prove that $C(x,c,N_1) \geq \delta_1^{opt}$ (if so, we can say $\delta_2^{\text{vhagar}} \leq  \delta_1^{opt}+\delta_{\text{diff}}+\Delta$ and then $\delta_2^{\text{vhagar}} \leq \delta_2^{opt}$). Lets' examine:
    $$|C(x,c,N_2)-C(x,c,N_1)| \leq \delta_{\text{diff}}$$
    $$ -\delta_2^{\text{vhagar}} \leq C(x,c,N_2)-C(x,c,N_1) \leq \delta_2^{\text{vhagar}} $$
    Specifically, we focus on $\delta_1^{opt}+\delta_{diff}+\Delta \leq C(x,c,N_2)$:
    $$  -\delta_2^{\text{vhagar}} \leq \delta_1^{opt}+\delta_{diff}+\Delta-C(x,c,N_1) \leq \delta_2^{\text{vhagar}} $$
    $$  0 \leq -\delta_1^{opt}-\Delta+C(x,c,N_1) \leq 2\cdot\delta_2^{\text{vhagar}} $$
    $$ \delta_1^{opt} \leq \delta_1^{opt} + \Delta \leq C(x,c,N_1) $$
    Therefore, we can say:
    $$I\{x| C(x,c,N_1)\geq\delta_1^{opt} + \Delta,\ C(x,c,N_2)\geq \delta_1^{opt}+\delta_{\text{diff}}+\Delta\}\ne\emptyset$$
    $$\text{max}_{x\in{I}}{C(x,c,N_2)} \geq \delta_2^{vhagar}$$
    which means that $\delta_2^{opt} \geq \delta_2^{vhagar}$
    
    \item If $\delta_1^{opt}+\delta_{\text{diff}}+\Delta > \delta_2^{max}$
    Again, we examine $\forall{x}:\ \delta_1^{opt} \leq C(x,c,N_1)$. In that case:
    $$|C(x,c,N_2)-C(x,c,N_1)| \leq \delta_{\text{diff}}$$
    $$-\delta_{\text{diff}} \leq C(x,c,N_2)-C(x,c,N_1) \leq \delta_{\text{diff}}$$
    $$-\delta_{\text{diff}}+C(x,c,N_1) \leq C(x,c,N_2) \leq \delta_{\text{diff}}+C(x,c,N_1)$$
    Applying $\delta_1^{opt} \leq C(x,c,N_1)$:
    $$\delta_1^{opt}-\delta_{\text{diff}} \leq C(x,c,N_2)$$
    
    \begin{enumerate}[nosep,nolistsep]
    \item If $\delta_2^{max} < \delta_1^{opt} - \delta_\text{diff}$:
    $$ \delta_2^{max}+\delta_\text{diff} < \delta_1^{opt}$$
    therefore:
    $$ \delta_2^{max}+\delta_\text{diff} - \delta_{\text{diff}}< \delta_1^{opt}-\delta_{\text{diff}} \leq C(x,c,N_2) $$
    $$ \delta_2^{max} < C(x,c,N_2) $$
    Which is not possible/contradiction.
    
    \item we get $\delta_1^{opt}-\delta_{\text{diff}} \leq \delta_2^{max}$ which means:
    $$ \exists{x}:\in \{C(x,c,N_2) \geq \delta_1^{opt}-\delta_\text{diff},\ C(x,c,N_1) \geq \delta_1^{opt}\} = I $$
    We can say that:
    $$\forall{x}:\ C(x,c,N_2)\geq \delta_2^{vhagar}+\Delta,\ C(x,c,N_1) \geq \delta_1^{opt} \rightarrow x\in{I}$$
    which says:
    $$ max_{x\in{I}}{C(x,c,N_2)} \geq \delta_2^{vhagar}+\Delta $$
    Which also means that $\delta_2^{opt} \geq \delta_2^{vhagar}+\Delta \geq \delta_2^{vhagar}$
    
    \end{enumerate} 
    
\end{enumerate} 



\begin{comment}
Let's examine the following optimization problem:
$$ max_x{\delta_\text{diff}}:\ 0 \leq |C(x,c,N_2)-C(x,c,N_1)|=\delta_\text{diff} $$
Which means that $\forall{x}:\ |C(x,c,N_2)-C(x,c,N_1)| \leq \delta_\text{diff}$.\\ \\
If $\delta_\text{diff}=0$ it means that $\forall{x}:\ |C(x,c,N_2)-C(x,c,N_1)|=0$, and specifically $\delta_1^{opt}=\delta_2^{opt}$. From previous lemma we can say $\delta_1^{opt} \geq \delta^{vhagar}_1$, so in that case, $\delta_2^{opt} \geq \delta^{vhagar}_1$. because of $\forall{x}:\ |C(x,c,N_2)-C(x,c,N_1)|=0$, we can say $\delta^{vhagar}_1=\delta^{vhagar}_2$ (otherwise it contradicts vaghar or the equality of the confidence-diff to 0). therefore: $\delta_2^{opt} \geq \delta^{vhagar}_2$ when $\delta_\text{diff}=0$.\\
We shall use the following notion: when $\delta_\text{diff}=0: \delta'=\delta_2^{opt}$ \\ \\
If $\delta_\text{diff}>0$: since $\forall{x}:\ |C(x,c,N_2)-C(x,c,N_1)| \leq \delta_\text{diff}$ it also happens:  
$$\forall{x}\in{\{x| C(x,c,N_2)\geq{\delta'},\ C(x,c,N_1)\leq{\delta_1^{opt}}\}}: |C(x,c,N_2)-C(x,c,N_1)| \leq \delta_\text{diff}$$

reminder: $\delta'=\delta_1^{opt}$. \\
therefore, from looking at $|C(x,c,N_2)-C(x,c,N_1)|\leq \delta_{\text{diff}}$ we can assume $C(x,c,N_2) \geq \delta' \geq \delta_2^{vaghar}$ when $C(x,c,N_2)-C(x,c,N_1) \geq 0$.
That means: $max_{x\in{I^{\delta'}}}\{C(x,c,N_2)\}\geq \delta'$ which means $\delta_2^{opt} \geq \delta' \geq \delta_2^{vhagar}$. \\
When $C(x,c,N_2)-C(x,c,N_1) <0$ we get $0<C(x,c,N_1)-C(x,c,N_2) \leq \delta_{\text{diff}}$
we can also define:
$$ I^{\delta'}=\{ x|C(x,c,N_2) \geq \delta'+\Delta,\ C(x,c,N_1) \geq \delta_1^{opt}+\Delta= \delta'+\Delta\} $$
when $x\in{I^{\delta'}}$ this case supposed to hold. therefore when $x\in{I^{\delta'}}$ we get $C(x,c,N_1) \geq \delta'+\Delta$.
Overall:
$$0 < \delta'+\Delta-C(x,c,N_2) \leq C(x,c,N_1)-C(x,c,N_2) \leq \delta_{\text{diff}}$$
Notice: $0 < \delta'+\Delta-C(x,c,N_2)$ is because otherwise we are in the first scenario.\\
this means $C(x,c,N_2) <\delta'+\Delta$ so $x\notin{I^{\delta'}}$ which is a contradiction. \\
this 




\begin{comment}
We can say that  $I_{\Delta}=\{x| C(x,c,N_2)\geq{{\delta_2^{opt}}_{\Delta=0}}\}\ \bigwedge\ \{x| C(x,c,N_1)\leq{\delta_1^{opt}}\}$. \\
However, the second group ($\{x| C(x,c,N_1)\leq{\delta_1^{opt}}\}$) is the same for $I_{\Delta>0}$ and for $I_{\Delta=0}$. Therefore:
$$ I_{\Delta\geq 0}^2=\{x| C(x,c,N_2)\geq{{\delta_2^{opt}}_{\Delta=0}}\}_{\Delta>0}  \supset I_{\Delta=0}^2=\{x| C(x,c,N_2)\geq{{\delta_2^{opt}}_{\Delta=0}}\}_{\Delta=0}$$
Which means that $\forall{x}\in{I_{\Delta\geq 0}^2-I_{\Delta=0}^2}: C(x,c,N_2) \geq C$
\end{comment}
\end{proof}

%\end{document} 